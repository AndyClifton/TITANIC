% !TEX root = UnofficialGuideToNWTC135mData.tex
\chapter{Low-frequency (10-minute) time-series data files}
The ten-minute summary data described in Chapter \ref{s:TenMinuteMatFiles} can be concatenated together to create a low-frequency time-series. As with the high-frequency data, this data may be available in a MATLAB format or as ASCII.

\section{MATLAB files\label{s:LFMatlab}}
A MATLAB data file will be created that contains a single structure \mvar{all\_data}. The fields in the structure will be the same as the fields in the 10-minute summary files (see Chapter \ref{s:TenMinuteMatFiles}), but will be arrays of the data collected in each 10-minute interval. For example:
\begin{itemize}
\item \mvar{all\_data.Raw\_<Channel\_name>\_<z>m\_mean}
\begin{itemize}
\item \mvar{\ldots\_mean.val} Array of mean values during each 10-minute interval, after removing data outside of manufacturer's limits.
\item \mvar{\ldots\_mean.date}. Array of serial date numbers of the start of the 10-minute interval.
\item \mvar{\ldots\_mean.label}. Text string to use as a label for charts, etc.
\item \mvar{\ldots\_mean.units}. Text string containing a \LaTeX-formatted string describing the units, e.g. \verb+'m s^{-1}'+.
\item \mvar{\ldots\_mean.height}. The height $z$ above ground [m].
\item \mvar{\ldots\_mean.npoints}. Array of the number of points in each 10-minute interval.
\item \mvar{\ldots\_mean.flags}. Cell array of QC codes.
\end{itemize}
\end{itemize}

There is another important variable in this file:
\begin{itemize}
\item \mvar{all\_data.version}
\begin{itemize}
\item \mvar{.version.val} Array of the code version that was used to prepare the data from this time interval. 
\end{itemize}
\end{itemize}

\subsection{Example: checking the version of the data}
Task: Check which version of the data processing software were used to create the data.

Find the unique values in the \mvar{all\_data.version} array. The values in this array should be identical:
\begin{lstlisting}
>> unique(all_data.version.val)
1.21
\end{lstlisting}

\subsection{Example: plotting the time series of wind speed}
Task: plot the wind speed at 80 m above ground against time. 

First, start by checking the variables in the workspace, remembering that the data is delivered as a large structure:

\begin{lstlisting}
>> who
Your variables are:

M4      tend    tstart 
>> M4
\end{lstlisting}

Note the name of the variable for horizontal wind speed at 80 m above ground. Assume for this example that it is \mvar{M4.Wind\_Speed\_Cup\_80m}. Next, plot the value versus time:

\begin{lstlisting}
figure
plot(M4.Wind_Speed_Cup_80m.date, M4.Wind_Speed_Cup_80m.val,'k-')
datetick('x')
xlabel('Date')
ylabel('Wind Speed [m/s]')
\end{lstlisting}

This plot includes all data, with no filtering for quality.

\subsection{Example: plotting only good data}
Continuing from the previous plot, we need to find data where the quality-control codes indicate good data. This means that the field, \mvar{M4.Wind\_Speed\_Cup\_80m.flag}, should be empty. Check the cells:

\begin{lstlisting}
ipass = false(size(M4.Wind_Speed_Cup_80m.val));

for i = 1:numel(ipass)
    ipass(i) = isempty(M4.Wind_Speed_Cup_80m.flags{i});
end
\end{lstlisting}

Now, use the logical array to plot a subset of the data that pass the quality tests. This should be plotted on top of the plot from the  previous example: 

\begin{lstlisting}
hold on
plot(M4.Wind_Speed_Cup_80m.date(ipass), ...
    M4.Wind_Speed_Cup_80m.val(ipass),...
    'go')
\end{lstlisting}


\subsection{Example: finding a subset of data}
Task: find all data where the wind speed is from a certain direction.

We need to filter all data to focus on winds from a range of directions. We'll do this first by finding the indices of the wind from the direction we want:

\begin{lstlisting}
WD =M4.Wind_Direction_Vane_88m_mean.val;
idir = logical((WD > 135) | (WD < 75));
isubset = idir & ipass;
\end{lstlisting}

Next, we loop through all of the variables to find the data we want.

\begin{lstlisting}
M4Fields = fieldnames(M4);

for fi = 1:numel(M4Fields)
    if isfield(M4.(M4Fields{fi}),'height')
        M4Subset.(M4Fields{fi}).label = M4.(M4Fields{fi}).label;
        M4Subset.(M4Fields{fi}).units = M4.(M4Fields{fi}).units;
        M4Subset.(M4Fields{fi}).height = M4.(M4Fields{fi}).height;
        M4Subset.(M4Fields{fi}).val = M4.(M4Fields{fi}).val(isubset);
        M4Subset.(M4Fields{fi}).npoints = M4.(M4Fields{fi}).npoints(isubset);
        M4Subset.(M4Fields{fi}).date = M4.(M4Fields{fi}).date(isubset);
        M4Subset.(M4Fields{fi}).flags = M4.(M4Fields{fi}).flags{isubset};
    end
end
\end{lstlisting}

And the structure \mvar{M4Subset} now contains all of the data that passes our selection criteria.

%%%%%%%%
%% ASCII %%
%%%%%%%%
\newpage
\section{ASCII files\label{s:LFASCII}}
The ASCII concatenated 10-minute summary data file has 2 header lines and then data lines. 

The header files are:
\begin{enumerate}
\item Variable names
\item Units
\end{enumerate}

The data lines are comma-separated, fixed width fields. Each row includes the values for a different 10-minute interval. Each column is a different variable. For a description of the variables that are included, see the description of the data products in Chapter \ref{s:TenMinuteMatFiles}.  Quality-control codes are also used, but have been simplified in the text file compared to the MATLAB data. Quality-control codes are included in the variable, \mvar{<var\_name>\_QC}. The following codes are used:
\begin{itemize}
\litem{1 = pass}
\litem{0 = flag}
\litem{-1  = fail}
\end{itemize}

Missing data are given the value -999. The reader is requested to confirm the data format (e.g. integer or $n$-digit precision) before filtering to remove missing data.